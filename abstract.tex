%!TEX root = thesis.tex

%%%%%%%%%%%%%%%%%%%%%%%%%%%%%%%%%%%%%%%%%%%%%%%%%%%%%%%%%%%%%
% abstract.tex: Abstract
%%%%%%%%%%%%%%%%%%%%%%%%%%%%%%%%%%%%%%%%%%%%%%%%%%%%%%%%%%%%%


%The abstract for this chapter is now a ``quote'' environment in which can use centering
Quantifying galaxy morphology is a challenging yet scientifically rewarding task. 
As the scale of data continues to increase with upcoming surveys, traditional 
classification methods will struggle to handle the load. We present a solution 
through an integration of visual and automated classifications, preserving the 
best features of both human and machine. We demonstrate the effectiveness of 
such a system through a re-analysis of visual galaxy morphology classifications 
collected during the Galaxy Zoo 2 (GZ2) project.
We reprocess the top-level question of the GZ2 decision tree
with a Bayesian classification aggregation algorithm dubbed SWAP, originally developed 
for the Space Warps gravitational lens project. 
Through a simple binary classification scheme we increase the classification rate nearly 5-fold classifying 226,124 galaxies in 92 days of GZ2 project time while reproducing labels derived from GZ2 classification data with 95.7\% accuracy.

We next combine this with a Random Forest machine 
learning algorithm that learns on a suite of non-parametric morphology indicators 
widely used for automated morphologies. We develop a decision engine that 
delegates tasks between human and machine and demonstrate that the combined 
system provides a factor of 11.4 increase in the classification rate, classifying 
210,543 galaxies in just 32 days of GZ2 project time with 93.5\% accuracy. As the Random Forest algorithm requires a minimal amount of computational cost, this result has
important implications for galaxy morphology identification tasks in the 
era of \textit{Euclid} and other large-scale surveys.











