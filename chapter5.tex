%!TEX root = thesis.tex

\chapter{A population of ``clumpy'' galaxies in local Universe}
\label{chap:5}


All the things about clumps. 

\section{Sample Selection \& Data} 
\label{sec:style}
What was put into Galaxy Zoo: Hubble? All of Stripe 82 (but what does that mean?)
Why were they put in? 

From Willett+17: \\
Single-epoch images from SDSS Stripe 82 were selected using the criteria from Willett et al. 2013, 
which required limits of \texttt{petroR90\_r}~$ > 3''$ and a magnitude brighter than $m_r < 17.77$. 21,522 galaxies in SDSS met these criteria. Co-added images from Stripe 82 were selected from the union of galaxies with co-added magnitudes brighter than 17.77 mag, and the galaxies detected in the stripe-82-single images and matched to a co-add source. This resulted in a total set of 30,339 images. Of the images in the co-added sample, 5144 (17\%) were dimmer than the initial cut of 17.77  mag.


Delicious side-effect: Clumpies!

We consider the Stripe 82 subjects that were part of the GZH project.  
The GZH catalog doesn't provide debiased labels for the Stripe 82 subjects, most likely because this has already been done before in GZ2 and because the debiasing is completely different for these low redshift subjects. If we're now going to pick out subjects that are ``featured" -- shouldn't we use the debiased GZ2 values? How much does this change our sample? For that matter, how similar are the smooth and featured vote fractions between GZ2 and GZH? 
\textbf{Not going to answer this question in this paper.}



To select a sample of ``clumpy" galaxies from the GZH Stripe 82 sample, we consider only those subjects with  high featured and high clumpy vote fractions; the fraction of volunteers who classified each subject as being featured or clumpy. Specifically, we began with a selection criteria of \ffeat $\ge 0.5$ and  \fclump $\ge 0.5$ and $N_{\mathrm{votes}} \ge 20$, where $N_{votes}$ is the number of volunteers who answerd the question ``Does the galaxy have a mostly clumpy appearance?".  This produced a a sample of 629 galaxies: 273 with single-epoch imaging and 356 with coadd imaging. After visual inspection we find that this is hardly a pure sample of traditional clumpy galaxies instead including many small groups of elliptical galaxies as well as galaxies in various merging states and possessing multiple nuclei. [\textbf{show example image?}] After excluding these and duplicate imaging, we retain 90 coadd-depth clumpy galaxies and 102 single-depth clumpy galaxies. Of these, 36 subjects have both single- and coadd-depth imaging.

 [Note: technically ALL of these galaxies have single- and coadd-depth imaging because they are all in Stripe82. it's only a point of contention here because not all of them have the GZH morphologies assigned to them, for various reasons.]


We next select only those subjects which have spectroscopic redshift < 0.06. Beyond this distance, the physical scale as observed with SDSS imaging is no longer similiar to Hubble's at z $\sim3$. The physical scale at z= 0.06 is 1"=1.1kpc. SDSS pixel scale is 0.396"/pixel. --> 0.43 kpc/pixel -->  ~2.3 pixels = 1kpc.   Our final sample contains 105 unique galaxies, each with at least one SDSS spectrum. Half of our sample consist of objects with multiple spectra. We visually inspect these to verify that the SDSS fiber was indeed positioned over a star-forming region rather than the galactic bulge or other structure. 

We obtain SDSS DR12 \textit{ugriz} coadd imaging and spectra within 30" for every galaxy in our sample, discarding those spectra which belong to nearby sources or stars. We find that one ``clumpy" galaxy is, in fact, a juxtoposition of three galaxies at disparate redshifts flagged as clumpy due to the low resolution of SDSS imaging. We exclude this subject(s) from our sample.  Our final list includes 175 spectra. Approximately half of the galaxies in our sample have more than one SDSS spectrum with a handful having three or four spectra.  



\section{Analysis}
During visual inspection we discover that several of the galaxies in our sample have very low surface brightness. Additionally, extremely bright clumps and nearby sources make SDSS photometry (and, subsequently, stellar masses) suspect. In this section we describe our analysis of basic galaxy parameters. 

We create postage stamps of each galaxy from each field and all bands of our SDSS coadd imaging. The cutout sizes are determined to be 5 times the petrosian radius as computed by the SDSS pipeline. As mentioned above, however, this pipeline fails to properly account for galaxy extent and thus produces unrealistic measurements of the petrosian radius. We visually inspect all of our cutouts and choose an appropriate postage stamp size that fully encompasses each galaxy. The postage stamp radius is determine from the r-band petrosian radius and this value is used as the stamp size for each band. 

We next process the r-band (maybe g-band cuz that's more blue?) postage stamp with Source Extractor (Bertin and whoever) using parameters designed to detect low surface brightness features. Important parameters are detailed in Table X that doesn't exist yet. Though these parameters adequately identify most of our sample, galaxies that fail are redone individually and SE parameters are tweaked for each particular case. These segmentation maps define the galaxy extent for all future analysis. 



\subsection{Measuring clump radial distance} \label{subsec:radii}


\subsection{Clump Halpha Hbeta ratios [Dust]} \label{subsec:dust}

\subsection{Clumpy Halpha equivalent widths} \label{subsec:ew}




%% If you wish to include an acknowledgments section in your paper,
%% separate it off from the body of the text using the \acknowledgments
%% command.

\section{Discussion / Conclusions}
our clumps are the best clumps. Everybody says so. 


We thank all the people who contributed to Galaxy Zoo. 