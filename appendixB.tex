
%% ----------------------------------------------------------------------------------------------------------------------------------------------
%% 		MEASURING MORPH PARAMS ON SDSS STAMPS
%% ----------------------------------------------------------------------------------------------------------------------------------------------

\chapter{Measuring Nonparametric Morphological Diagnostics on SDSS Stamps} 
\label{sec: measuring morphology}


In order to train our Random Forest machine learning algorithm, we measure  non-parametric
morphology diagnostics for the GZ2 galaxy sample. 

We obtain $i$-band imaging from SDSS Data Release 12. Postage stamps are made from 
the SDSS fields for each galaxy with dimensions of 3 Petrosian radii. Galaxies located 
within 3 Petrosian radii of the edge of a field were excluded.  Postage stamps undergo a
 cleaning process whereby nearby sources are identified with SExtractor \citep[ver. 2.8.6;][]{sextractor} and their pixels replaced with values that mimic the background in that region. 
We compute the following widely adopted nonparametric measurements  of the galaxy light distribution on the cleaned postage stamps:

Concentration is computed as $C = 5\log(r_{80}/ r_{20})$ where \rr{80} and \rr{20} are the
radii containing 80\% and 20\% of the galaxy light respectively.  Small values of this ratio 
tend to indicate disky galaxies, while larger values correlate with early-type ellipticals. 

Asymmetry quantifies the degree of rotational symmetry in the galaxy light distribution
 (not necessarily the physical shape of the galaxy as this parameter is not highly sensitive 
to low surface brightness features). A correction for background noise is applied (as in e.g.~\cite{Conselice2000}), i.e., 
\begin{equation}
A = \frac{\sum_{x,y} |I - I_{180}|}{ 2\sum|I|} - B_{180}
\end{equation}
where $I$ is the galaxy flux in each pixel $(x, y)$, $I_{180}$ is the image rotated by 180 degrees about the galaxy's central pixel, and $B_{180}$ is the average asymmetry of the background. 

The Gini coefficient, $G$,~\citep{Glasser1962, Abraham2003} describes how uniformly distributed a galaxy's flux is.  If $G$ is 0, the flux is distributed homogeneously among all galaxy pixels; if $G$ is 1,  the light is contained within a single pixel. This term correlates with $C$, however, $G$ does not require that the flux be in the central region of the galaxy.  We follow~\cite{Lotz2004} by first ordering the pixels by increasing flux value, and then computing
\begin{equation}
G = \frac{1}{|\bar X|n(n-1)}\sum_i^n(2i-n-1)|X_i|
\end{equation}
where $n$ is the number of pixels assigned to the galaxy, and $\bar X$ is the mean pixel value. 

\M{20}~\citep{Lotz2004} is the second order moment of the brightest 20\% of the galaxy flux. We compute it as
\begin{eqnarray}
 M_{tot} & = & \sum_i^nf_i[(x_i-x_c)^2 + (y_i-y_c)^2]  \\
 M_{20} & = & \log_{10} (\frac{\sum_iM_i}{M_{tot}}), ~~\textrm{while} \sum_ifi < 0.2f_{tot}
\end{eqnarray}
where M$_{tot}$, the total moment, is computed first and $f_{tot}$ is the total flux. For centrally concentrated objects, \M{20} correlates with $C$ but is also sensitive to bright off-centre knots of light. 

Finally, we use the ellipticity, $\epsilon = 1 - b/a$, of the light distribution as measured by SExtractor which computes the semi-major axis $a$ and semi-minor axis $b$ from the second-order moments of the galaxy light.  

In total, we measure morphological indicators for 282,350 SDSS galaxies. The relations between these diagnostics for the full sample is shown in the right panel of Figure~\ref{fig: morph thresh}. The code developed to clean and compute these morphology indicators is open source and can be found at \url{https://github.com/melaniebeck/measure_morphology}.


