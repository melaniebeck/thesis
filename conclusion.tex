%%%%%%%%%%%%%%%%%%%%%%%%%%%%%%%%%%%%%%%%%%%%%%%%%%%%%%%%%%%%%%%%%%%%
%!TEX root = thesis.tex
%%%%%%%%%%%%%%%%%%%%%%%%%%%%%%%%%%%%%%%%%%%%%%%%%%%%%%%%%%%%%%%%%%%%
\chapter{Summary \& Future Work}
\label{chap:summary}

The goal of this thesis was to develop a framework that brings both human and machine intelligence to the task of galaxy morphology to handle the scale and scope of next generation of astrophyical surveys. To this end we have measured traditional automated morphologies for $\sim$280K SDSS galaxies to use as features in a machine learning algorithm. We also utilized original Galaxy Zoo 2 volunteer classifications in various simulations in order to explore the impact of intelligent retirment of visual galaxy morphology classifications of a crowd-sourced project.  

Our simulations of Galaxy Zoo 2 classifications utilized 





We demonstrate nearly a factor of five increase in the classification rate, a reduction of
at least a factor of three in the human effort necessary to maintain that increased rate,
all while maintaining 95\% accuracy, nearly perfect completeness of ‘Featured’ subjects,
and with a purity that can be controlled by careful selection of input parameters to be
better than 90\% (see Appendix 3.4). Exploring those subjects wherein SWAP and GZ2
disagree, we conclude that the majority of this disagreement stems from the stochastic
nature of GZ2 raw labels. We now turn our focus towards incorporating a machine
classifier utilizing these SWAP-retired subjects as a training sample.

The goal of this thesis was to study the mass-loss histories of hypergiants stars using new capabilities in near-IR imaging and polarimetry, and airborne mid-IR imaging.  With LBT/LMIRCam and MMT-Pol on the MMT we have imaged the nebulae of VY CMa and IRC +10420 with sub-arcsecond resolution, revealing recent mass-loss in the close environments around these stars.  To probe further into these hypergiants' past history, we have used mid-infrared imaging with SOFIA/FORCAST to search for cold dust and performed 1-D radiative transfer modeling of their SEDs and resolved profiles.

Our 2 $-$ 5 $\micron$ adaptive optics imaging of the cool hypergiant VY CMa penetrates deeper into its dusty nebula than in the optical, probing its recent mass-loss history.  In Chapter 2 we analyzed the resolved images of its peculiar ``Southwest'' Clump, which has no obvious counterpart on the opposite side of the star.   The distinct shape of the SW Clump is suggestive of a short-lived, localized event and may be analogous to a coronal mass ejection (CME) from a single location on the Sun's surface.  A short-lived ejection event is consistent with the SW Clump appearing as a confined, coherent shape several hundred years after ejection.  Using adaptive optics imaging polarimetry, in Chapter 3 we demonstrated the Clump is optically thick through at least 3.1 $\micron$ and reaffirmed the lower limit mass of 5 $\times$ 10$^{-3}$ $M_{\sun}$ based on modeling its surface brightness as optically thick scattered light.  Our 1.3 $\micron$ polarimetry detects several other prominent features of VY CMa's nebula include the NW Arc, Arc 2, S Knot, and S Arc.  Using the polarized intensity as a lower limit on total scattered light intensity, we found each of these features to be optically thick as well.  Their relatively high intrinsic polarizations are consistent with their high scattering optical depths since the depolarizing effect of multiple scatters is reduced for typical silicate grain albedos.  In Chapter 4, we used 20 $-$ 37 $
\micron$ infrared imaging with SOFIA/FORCAST to search for evidence of earlier mass loss.  VY CMa's morphology at the longest wavelengths coincides with the general shape of the highly asymmetric nebulae seen in the visual, suggesting thermal emission from dust associated with the expanding arcs to the northwest and southwest.  Modeling its SED we computed an average mass-loss rate of 6 $\times$ 10$^{-4}$ $M_{\sun}$ yr$^{-1}$ over the past $\sim$1200 years, with no clear evidence of mass loss much farther in its past.

Our study of the warm hypergiant IRC +10420 similarly traced its mass-loss over a range of angular scales.  At the sub-arcsecond scale using adaptive optics, in Chapter 3 we used 2.2 $\micron$ polarimetry to reveal a relatively uniform nebula largely in the plane of the sky extending out to 2$\farcs$5 from the star.  This low-latitude ejecta is optically thick.  Combining the polarimetry with 3 $-$ 5 $\micron$ imaging that shows extended emission, we modeled the flux of this nebula and found its emission is an order of magnitude brighter than can be explained by simple extrapolation of the scattered light seen at $2.2~\micron$.  We hypothesized grains warmed to a temperature higher than the expected grain equilibrium temperature, but consistent with the local gas temperature in this region.  In Chapter 4 we presented 8 $-$ 12 $\micron$ adaptive optics images that reveal spatially extended emission spanning nearly the same range as the 2.2 $\micron$ polarimetry.  Applying radiative transfer modeling to the intensity profile of this extended emission and the SED, we found that IRC +10420's mass-loss history is divided into two distinct periods.  Our best-fit model showed that it lost mass at a high average rate of 2 $\times$ 10$^{-3}$ $M_{\sun}$ yr$^{-1}$ from 6000 $-$ 2000 yr ago during its presumed RSG stage, followed by an order of magnitude decrease to an average rate of 1 $\times$ 10$^{-4}$ $M_{\sun}$ yr$^{-1}$ in the past 2000 yr.

In addition to VY CMa and IRC +10420, our SOFIA/FORCAST program included the RSG $\mu$ Ceph and the warm hypergiant $\rho$ Cas.  Our study of $\mu$ Cep probes a mass-loss history extending back $\sim$13,000 yr.  To match the observed intensity profile of its resolved nebula in FORCAST 25 $-$ 37 $\micron$ images, our radiative transfer modeling requires a declining mass-loss rate.  We find that over the 13,000 yr dynamical age of the shell, its mass loss-rate has declined from 5 $\times$ 10$^{-6}$ to 1 $\times$ 10$^{-6}$ $M_{\sun}$ yr$^{-1}$.  In contrast to the high mass-loss rate of VY CMa, $\mu$ Cep's rate is significantly lower than RSGs of comparable luminosity.  In the case of $\rho$ Cas, we demonstrated that our new 19.7 $-$ 37.1 $\micron$ SOFIA/FORCAST photometry are consistent with the continued expansion and thinning of the dust shell formed as a result of the 1946 eruption.  We did not find evidence of new dust formation from more recent eruptions.

In the future, infrared imaging over a range of angular scales will be used to expand the sample of RSGs and post-RSGs with resolved mass-loss.  The SOFIA/FORCAST Cycle 3 program 03\_0082 (PI: R. Humphreys) included three additional RSG targets (NML Cyg, VX Sgr and S Per) and contingent approval has been granted for observations of the yellow hypergiant HR 5171A during a Cycle 4 Southern deployment if flight scheduling permits.  In Fall 2016, VY CMa is scheduled for observation with NOMIC, the 8 $-$ 13 $\micron$ imager on the Large Binocular Telescope.  With NOMIC's 0$\farcs$27 angular resolution (for single dish observing mode), VY CMa's SW Clump should be resolved.  These observations will better sample the Clump's SED and could help resolve the puzzle of its non-detection by ALMA reported in \citet{OGorman:2015}.  

The work presented in this thesis has demonstrated the capabilities of MMT-Pol and LMIRCam to resolve circumstellar ejecta around RSGs and post-RSGs.  These capabilities will be extended to several recently identified RSG clusters, each of which contains substantial coeval RSG populations.  These include RSGC1 \citep{Figer:2006,Davies:2008}, RSGC2 $=$ Stephenson 2 \citep{Davies:2007}, NGC 7419 \citep{Marco:2013} and the Per OB1 association, which contains many RSGs in its halo including S Per \citep{Humphreys:1978}.  Imaging these clusters of RSGs in polarized intensity in the near-infrared with MMT-Pol will cleanly separate light scattered by dusty nebulae from the stars' light, and when combined with adaptive optics imaging from 3 $-$ 13 $\micron$ with LMIRCam/NOMIC will enable us to trace the RSGs mass-loss.  Since the RSG cluster populations are coeval, this will allow the role of high mass loss events on their evolution to be studied.








