%%%%%%%%%%%%%%%%%%%%%%%%%%%%%%%%%%%%%%%%%%%%%%%%%%%%%%%%%%%%%%%%%%%%
%!TEX root = thesis.tex
%%%%%%%%%%%%%%%%%%%%%%%%%%%%%%%%%%%%%%%%%%%%%%%%%%%%%%%%%%%%%%%%%%%%
\chapter{Summary \& Future Work}
\label{chap:summary}

The goal of this thesis was to develop a framework that brings both human and machine intelligence to the task of galaxy morphology classifications to handle the scale and scope of next generation of astrophyical surveys. We demonstrate the effectiveness of our solution, Galaxy Zoo: Express (GZX), through a re-analysis of visual galaxy morphology classifications collected during the Galaxy Zoo 2 crowd-sourced project, and combine these with a Random Forest machine learning algorithm that trains on a suite of non-parametric morphology indicators widely used for automated morphologies. 

We measure traditional automated morphologies for $\sim$280K SDSS galaxies to use as features in our Random Forest algorithm. Specifically we measure the concentration index, asymmetry, \M{20}, Gini coefficient, and ellipticity; parameters that are well known to correlate strongly with the distinction between early- and late-type galaxies. We present a catalog of all such measurements and demonstrate that they are generally robust against various possible failure mechanisms we explore.


We demonstrate that classification efficiency can be increased through intelligenit t management of visual classifications from volunteers. We examine this through several simulations whereby we re-process Galaxy Zoo 2 classifications with SWAP, a Bayesian code first developed for the Space Warps gravitational lens discovery project (Marshall et al., 2016). We demonstrate for the first time that the SWAP algorithm is robust for use in galaxy morphology classification. We show that by implementing SWAP on GZ2 classification data we can increase the rate of classification by a factor of 4-5, requiring only 90 days of GZ2 project time to classify nearly 80\% of the entire galaxy sample. Furthermore, we achieve a reduction of at least a factor of three in the human effort necessary to maintain that increased rate, all while maintaining 95\% classification accuracy, nearly perfect completeness of ‘Featured’ subjects, and with a purity that can be controlled by careful selection of input parameters to be better than 90\%. Exploring those subjects wherein SWAP and GZ2 disagree, we conclude that the majority of this disagreement stems from the stochastic nature of GZ2 vote fractions from which we assign binary labels. 

We implement and test a Random Forest algorithm and develop a decision engine that delegates tasks between human and machine. After a sufficient number of subjects have been classified by humans and processed through SWAP, the machine is trained and its performance assessed through cross-validation. We show that even this simple machine is capable of providing significant gains in the classification rate when combined with human classifiers: GZX retires over 70\% of GZ2 galaxies in just 32 days of GZ2 project time. This represents a factor of 11.4 increase in the classification rate as well as an order of magnitude reduction in human effort compared to the original GZ2 project. This is achieved without sacrificing the quality of classifications as we maintain accuracy well above 90\% throughout our simulations. Additionally, we have shown that training on a 5-dimensional parameter space of traditional non-parametric morphology indicators allows the machine to identify subjects that humans miss, providing a complementary approach to visual classification. The gain in classification speed allows us to tackle the massive amount of data promised from large surveys like LSST and Euclid.


GZX is perhaps one of the simplest ways to combine human and machine intelligence and it(s impressive performance motivates a higher level of sophistication. A first step will be an implementation of SWAP that can handle a complex decision tree. In addition, we envision multiple forms of active feedback in addition to our passive feedback mechanism. SWAP allows us to leverage the most skilled volunteers to review galaxies difficult for either human or machine to classify. Additionally, machine-retired subjects should contribute to the training sample for humans in an analogous fashion to what we have already implemented. Secondly, our RF can be improved by providing it information equal to what humans receive: multi-band morphology diagnostics will be included in our future feature vector. However, the Random Forest algorithm is not easily adapted to handle measurement errors or class labels with continuous distributions. To fully utilize the information provided by SWAP, sophisticated algorithms should be considered. The most likely candidate are deep convolutional neural networks (CNNs) such as those employed in recent studies with high levels of accuracy \citep{DominguezSanchez2017,HuertasCompany2008}. The drawback to most deep CNNa is the computational cost required to train such a model. Our Random Forest requires a fraction of the computational power as a CNN and fewer training samples as well, thus demonstrating that it should be considered as a vaiable appropach to morphology classification.  However, there is no reason to limit to a single machine. As hinted at in Figure \ref{fig: schematic}, several machines could train simultaneously, their predictions aggregated through SWAP, creating an on-the-fly machine ensemble.



Finally, we identify a sample of ``clumpy''vgalaxies in the local universe through the Galaxy Zoo: Hubble project which included imaging from Stripe 82 of the SDSS. We isolate 104 galaxies that have a traditional clumpy morphology and acquire SDSS imaging and spectroscopic data for these galaxies, including spectra for over 150 clumps. These galaxies have a median log stellar mass of 9.4, a sample that could provide crucial constraints on star formation models of low mass galaxies. We obtain stellar mass and star-formation rates for these galaxies through the GSWLC catalog \citep{Salim2016}. Our preliminary analysis of the spectra reveals that the star-forming regions in these galaxies are in many ways similar to high-redshift clumps: our clumps have a median physical size of $\sim1$kpc and median \ha~velocity dispersion of $\sim40$km/s, values that are only slightly lower than high redshift clumps. We also compare several spectroscopically derived properties with clump galactocentric radius but find no strong correlations.

These findings motivate not only a more in depth analysis, but also justify the search for similar galaxies in the local universe. To that end we described the Clump Scout project designed to search for additional clumpy galaxies in the local universe by presenting color-composite SDSS non-Stripe 82 imaging to the general public. We have made preliminary tests of this project with highly favorable reviews from volunteers and expect to find an additional 1000 clumpy galaxies, with upwards of $\sim$1500 associated spectra. This will provide a large statistical sample of local clumpy galaxies which we will use to conduct several additional studies for which we have recently recieved a NSF grant. In particular we will constrain the fraction of clumpy galaxies at low redshift, as well as investigate the age and metallicity of individual clumps as a function of galaxy stellar mass, galactocentric distance, and environmental density, allowing us to distinguish  between various clump formation theories.


%In the future, infrared imaging over a range of angular scales will be used to expand the sample of RSGs and post-RSGs with resolved mass-loss.  The SOFIA/FORCAST Cycle 3 program 03\_0082 (PI: R. Humphreys) included three additional RSG targets (NML Cyg, VX Sgr and S Per) and contingent approval has been granted for observations of the yellow hypergiant HR 5171A during a Cycle 4 Southern deployment if flight scheduling permits.  In Fall 2016, VY CMa is scheduled for observation with NOMIC, the 8 $-$ 13 $\micron$ imager on the Large Binocular Telescope.  With NOMIC's 0$\farcs$27 angular resolution (for single dish observing mode), VY CMa's SW Clump should be resolved.  These observations will better sample the Clump's SED and could help resolve the puzzle of its non-detection by ALMA reported in \citet{OGorman:2015}.  

%The work presented in this thesis has demonstrated the capabilities of MMT-Pol and LMIRCam to resolve circumstellar ejecta around RSGs and post-RSGs.  These capabilities will be extended to several recently identified RSG clusters, each of which contains substantial coeval RSG populations.  These include RSGC1 \citep{Figer:2006,Davies:2008}, RSGC2 $=$ Stephenson 2 \citep{Davies:2007}, NGC 7419 \citep{Marco:2013} and the Per OB1 association, which contains many RSGs in its halo including S Per \citep{Humphreys:1978}.  Imaging these clusters of RSGs in polarized intensity in the near-infrared with MMT-Pol will cleanly separate light scattered by dusty nebulae from the stars' light, and when combined with adaptive optics imaging from 3 $-$ 13 $\micron$ with LMIRCam/NOMIC will enable us to trace the RSGs mass-loss.  Since the RSG cluster populations are coeval, this will allow the role of high mass loss events on their evolution to be studied.