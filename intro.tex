%!TEX root = thesis.tex

%%%%%%%%%%%%%%%%%%%%%%%%%%%%%%%%
% intro.tex: Introduction to the thesis
%%%%%%%%%%%%%%%%%%%%%%%%%%%%%%%%
\chapter{Introduction}
\label{chap:intro}

Here we define the big overarching problem. \\
Why classify galaxies? \\
What classifictions are needed?\\
How do we get these classifications?\\
What problems do traditional methods face?\\

Chapter on a literature study of machine learning in astronomy? Or just in galaxy classification? And some machine learning basics. \\
Chapter that is essentially my paper but with an added section on the exploration of dimensionality reduction algorithms we explored. \\
Chapter on the science case of rare clumpy galaxies in the local universe found thanks to GZH.\\
Appendix: Stupid spectro-polarimetry paper\\


We're going to add in all the shit that I've done in the past three years -- much of that didn't make it into my paper. 
We can break up the work into two major parts: \\
1. SWAP + GZ 


BEEF THIS UP.\\
Astronomers have made use of visual galaxy morphologies to understand the dynamical structure of these systems for nearly ninety years 
\citep[e.g.,][]{Hubble1936, 
			deVauc1959, 
			Sandage1961, 
			vandenBergh1976, 
			NairAbraham2010, 
			Baillard2011}. 
The division between early-type and late-type systems corresponds, for example, to a wide range of parameters from mass and luminosity, to environment, colour, and star formation history 
\citep[e.g.,][]{Kormendy1977,  
			Dressler1980, 
			Strateva2001, 
			Blanton2003, 
			Kauffman2003, 
			Nakamura2003, 
			Shen2003, 
			Peng2010}; 
while detailed observations of morphological features such as bars and bulges 
provide information about the history of their host systems 
\citep[e.g., review by][]{KK04, 
			Elmegreen2008, 
			Sheth2008, 
			Masters2010, 
			Simmons2014}. 
Modern studies of morphology  divide systems into broad classes 
\citep[e.g.,][]{Conselice2006, 
			Lintott2008, 
			Kartaltepe2015, 
			Peth2016}, 
but a wealth of information can be gained from identifying new and often rare classes, 
such as low redshift clumpy galaxies \citep[e.g.,][]{Elmegreen2013}, polar-ring galaxies \citep[e.g.,][]{Whitmore1990}, and the green peas \citep{Cardamone2009}. 


While the Galaxy Zoo project has provided a solution that scales visual classification 
for current surveys  by harnessing the combined power of thousands of volunteers \citep{Lintott2008, Lintott2011, Willett2013, Willett2017, Simmons2017},
 producing a prolific amount of scientific output \citep[e.g.,][]{Land2008, Bamford2009,
 Darg2010, Schawinski2014, Galloway2015, Smethurst2016}; upcoming surveys such as
 \textit{LSST} and \textit{Euclid} will require a different approach, imaging more than
 a billion new galaxies  \citep{LSST, Euclid}.  If detailed morphologies can be extracted 
for just  0.1\% of this imaging, we will have millions of images to contend with. 
A project of this magnitude would take more than sixty years to classify at 
Galaxy Zoo's current rate and configuration. Standard visual morphology    
methods will thus be unable to cope with the scale of data. 

Another approach has been the automated extraction of morphologies with the
 development of parametric \citep{Sersic1968, Odewahn2002, Peng2002}, 
and non-parametric \citep{Abraham1994, Conselice2003, Abraham2003, Lotz2004,
 Freeman2013} structural indicators. While these scale well to large samples 
\citep[e.g.,][]{Simard2011, 
			Griffith2012, 
			Casteels2014, 
			Holwerda2014, 
			Meert2016}, 
they often fail to capture detailed structure and can provide only statistical morphologies with large uncertainties \cite[e.g.,][]{Abraham1996, Bershady2000}. 



Machine learning techniques are becoming increasingly popular for classification 
and image processing tasks. Another automated approach, these generally work
by defining a set of features that describe the morphology in an $N$-dimensional space. 
The location in this morphology space defines a morphological type for each galaxy.
Learning the morphology space can be achieved through algorithms such as 
Support Vector Machines \citep{HuertasCompany2008} 
or Principal Component Analysis \citep{Watanabe1985, Scarlata2007}.  
Another approach is through deep learning, a machine learning technique that attempts 
to model high level abstractions. Algorithms like convolutional and artificial 
neural networks (CNNs, ANNs) have been used for galaxy morphology classification 
with impressive accuracy 
\citep{Ball2004, 
	Banerji2010, 
	Dieleman2015, 
	HuertasCompany2015}. 
A drawback to all machine learning classification techniques is the need for 
standardized training data, with more complex algorithms requiring more data. 
Furthermore, these data must be consistent for each survey: differences in resolution 
and depth can be implicitly learned by the algorithm making their application to 
disparate surveys challenging.  

 In this work we present a system that preserves the best features of both visual 
and automatic classifications, developing for the first time a framework that 
brings both human and machine intelligence to the task of galaxy morphology to 
handle the scale and scope of next generation data. We demonstrate the 
effectiveness of such a system through a re-analysis of visual galaxy morphology
 classifications collected during the Galaxy Zoo 2 project, and combine these with a 
Random Forest machine learning algorithm that trains on a suite of non-parametric
 morphology indicators widely used for automated morphologies. 
In this paper we focus on the first question of the Galaxy Zoo 
decision tree. We demonstrate that our method 
provides a factor of 11.4 increase in the rate of galaxy morphology classification
 while maintaining at least 93.5\% classification accuracy as compared to Galaxy Zoo 2 published data. 
We first present an overview of our framework, which also serves as a blueprint for this paper. 


%%%-------------------------------------------------------
%%% FIGURE:     GZ EXPRESS Schematic
%%%-------------------------------------------------------
\begin{figure*}[ht!]
%\figurenum{1}
\plotone{Figures/human_machine/f1.pdf}
\caption[Schematic of the human+machine hybrid system.]{Schematic of our hybrid system. Humans provide classifications of galaxy images via a web interface. We simulate this with the Galaxy Zoo 2 classification data described in Section~\ref{sec: data}. Human classifications are processed with an algorithm described in Section~\ref{sec: SWAP}. Subjects that pass a set of thresholds are considered human-retired (fully classified) and provide the training sample for the machine classifier as described in Section~\ref{sec: machine}. The trained machine is applied to all subjects not yet retired. Those that pass an analogous set of machine-specific thresholds are considered machine-retired. The rest remain in the system to be classified by either human or machine. This procedure is repeated  nightly. Our results are reported in Section~\ref{sec: results}.  \label{fig: schematic}}
\end{figure*}


%%----------------------------------------------------------------------------------------------------------------------------------------------------
%%   GALAXY ZOO EXPRESS OVERVIEW
%%---------------------------------------------------------------------------------------------------------------------------------------------------
\section{Galaxy Zoo Express Overview}

The Galaxy Zoo Express (GZX) framework combines human and machine to
 increase morphological classification efficiency, both in terms of the classification rate 
and required human effort. Figure~\ref{fig: schematic} presents a schematic of 
GZX including section numbers as a shortcut for the reader. We note that transparent 
portions  of the schematic represent areas of future work which we explore in 
Section~\ref{sec: visions}. Any system combining human and machine classifications 
will have a set of generic features: a group of human classifiers, at least one 
machine classifier, and a decision engine which determines how these 
classifications should be combined.

In this work we demonstrate our system through a re-analysis of  Galaxy Zoo 2 (GZ2)
 classifications. This allows us to  create simulations of human classifiers (described in
 Section~\ref{sec: data}). These classifications are used most effectively when processed 
with SWAP, a Bayesian code described in Section~\ref{sec: SWAP}, first developed 
for the Space Warps gravitational lens discovery project~\citep{Marshall2016}. 
These subjects provide the machine's training sample. 

In Section~\ref{sec: machine}, we incorporate a machine classifier. We have 
developed a Random Forest algorithm that trains on measured morphology 
indicators such as Concentration, Asymmetry, Gini coefficient and 
\M{20} well-suited for the top-level question of the GZ2 decision tree, 
discussed below. After a sufficient number of subjects have been classified 
by humans,  the machine is trained and its performance assessed through 
cross-validation. This procedure is repeated nightly and the machine's performance 
increases with size of the training sample, albeit with a performance limit. 
Once the machine reaches an acceptable level of performance it is applied to the 
remaining galaxy sample. 

Even with this simple description, one can see that the classification process 
will progress in three phases.  First, the machine will not yet have reached an 
acceptable level of performance; only humans contribute to subject classification. 
Second, the machine's performance will improve; both humans and machine will 
be responsible for classification. Finally, machine performance will slow; 
remaining images will likely need to be classified by humans. These results are 
explored in  Section~\ref{sec: results}. This blueprint allows even modest 
machine learning routines to make significant contributions alongside human 
classifiers and removes the need for ever-increasing performance in machine classification.





