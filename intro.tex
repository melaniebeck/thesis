%!TEX root = thesis.tex

%%%%%%%%%%%%%%%%%%%%%%%%%%%%%%%%
% intro.tex: Introduction to the thesis
%%%%%%%%%%%%%%%%%%%%%%%%%%%%%%%%
\chapter{Introduction}
\label{chap:intro}


A galaxy's morphology is the culmination of its formation, interactions, and evolution through environmental as well as internal processes. It is a snapshot into the current state of a galaxy's life as well as window to its past. Though sometimes subtle, the insights gleaned through the study of galaxy morphology have radically changed our view of universe since the time of Edwin Hubble. 


Astronomers have made use of visual galaxy morphologies to understand the dynamical structure of these systems for nearly ninety years 
\citep[e.g.,][]{Hubble1936, 
			deVauc1959, 
			Sandage1961, 
			vandenBergh1976, 
			NairAbraham2010, 
			Baillard2011}. 
The division between early-type and late-type systems corresponds, for example, to a wide range of parameters from mass and luminosity, to environment, colour, and star formation history 
\citep[e.g.,][]{Kormendy1977,  
			Dressler1980, 
			Strateva2001, 
			Blanton2003, 
			Kauffman2003, 
			Nakamura2003, 
			Shen2003, 
			Peng2010}; 
while detailed observations of morphological features such as bars and bulges provide information about the history of their host systems 
\citep[e.g., review by][]{KK04, 
			Elmegreen2008, 
			Sheth2008, 
			Masters2010, 
			Simmons2014}. 
Modern studies of morphology  divide systems into broad classes 
\citep[e.g.,][]{Conselice2006, 
			Lintott2008, 
			Kartaltepe2015, 
			Peth2016}, 
but a wealth of information can be gained from identifying new and often rare classes, such as low redshift clumpy galaxies \citep[e.g.,][]{Elmegreen2013}, polar-ring galaxies \citep[e.g.,][]{Whitmore1990}, and the green peas \citep{Cardamone2009}. 

Obtaining these morphologies has traditionally been a time-consuming visual endeavor and only in the past twenty years have automated morphological assignment been possible. Even with the varied automated approaches currently being exploited, an era of Even Bigger Data looms for the field of astronomy. The next decade will herald the first light of more powerful ground- and spaced-based telescopes such as the Large Synaptic Survey Telescope (LSST), Euclid, and WFIRST. The surveys planned for these instruments promise to revolutionize the field of astrophysics providing several orders of magnitude more data than all of Everything? I don't know. Traditional techniques will not be sufficient for extracting galaxy morpholgoies on a pertinent timescale. 

This thesis details a solution to the scalability of galaxy morphology designations by examinging classification obtained as part of the Galaxy Zoo project, a crowd-sourcing initiative that has obtained morphological classifications from over a million volunteers for over a million galaxies from several surveys. Though innovative, even crowd-sourcing will be unable to sustain the classification load for future surveys. Instead, these classifications are combined with supervised machine learning algorithms that train on automated measurements of galactic morphological structural indicators. This thesis begins with a detailed account of the methodology used to obtain these morphological structural indicators (Chapter \ref{chap:2}). Chapter \ref{chap:3} demonstrates how crowd-sourcing techniques can be optimized by applying SWAP [come up with generic words] to Galaxy Zoo classifications, while Chapter \ref{chap:4} explores the combination of visual classifications with machine learning algorithms. Also included is a prelimary analysis of a rare sample of ``clumpy'' galaxies in the local universe discovered during the course of the analysis of Galaxy Zoo classifications [NOT GOOD WORDS] (Chapter \ref{chap:5}). While common in the distant universe, these potential local counterparts could shed light on  star-formation things and stuff. This Introduction provides a brief overview of galaxy formation, the sciece of morphology, and galaxy classification techniques, as well as an overview of Galaxy Zoo: Express, an integrated framework of human and machine galaxy classifiers. 




\section{Galaxy formation}
Brief overview of how galaxies form and the dizzying variety of morphologies. 


\secton{Standard morphological categorizations}
briefly discuss the Hubble tuning fork. Explain the difference between elliptical, spiral and in-between galaxies. As well as irregulars/peculiars. 


\section{Morphology as a tracer of galaxy evolution}
then segue into how those different morphologies provide detailed insights into formation and evolution. Give examples of the science that can be gleaned from broad classifications (early- vs late-type galaxies), and how rare classifications can provide unique viewpoints (clumpy galaxies at low redshift, or green peas, i.e.)

Brief review of why we care about morphology -- i.e., it correlates with all kinds of sciency bits so we should keep doing it. 

\subsection{Mass and luminosity}

\subsection{Morphology-density relation}
A relationship between morphology and the density in which the galaxy resides has been known for some time. It has been found that in the richest and densest clusters of galaxies, the dominant morphology is elliptical, while for field galaxies, the dominant morphology is spiral. Furthermore, it's been determined that this relation evolves over time, at least for the densest environments. All of this points to scenarios in which the environment of a galaxy dictates, or influences its morphology and that the processes contributing to this effect have evolved with time. \citep[e.g.,][]{Fasano2000, Shen2003, Smith2005, Peng2010}

Spirals convert to S0/elliptical galaxies in clusters due to several processes: they should have lost their cold atomic and molecular gas via ram pressure stripping (but see Tonnesen & Bryan 2009, for the effect of ram
pressure on molecular clouds), harassment (Moore et al. 1996),
strangulation (Kawata & Mulchaey 2008), etc.


\subsection{Color}

\subsection{Star formation history}

\subsection{Insights from rare morophologies}
What have we learned from the green peas? Clumpy galaxies at low redshift? Ring galaxies? 

-- insights concerning dynamical disks; star formation; blah. 


\section{Obtaining morphological designations}
Now that we all believe we should get these morphologies -- how should we do it? 

\subsection{Visual classifications}
The history of galaxy morphology assignment is rife with historical precendents due in large part to Edwin Hubble's original ``tuning fork.'' Seeing as Hubble originally thought these `nebulae' were features residing in our own galaxy, we end up with goofy shit where we call elliptcals ``early-type'' and disks ``late-type'' though subsequent studies all confirm that the ages of ellipticals are far older than those of spiral galaxies. 

Visual classification, though highly accurate due to the human mind's unique pattern recognition capability, is, however, entirely slow. Assignment of morphological type to galaxies thus resulted in small samples lacking statistical signifigance for decades (until the use of cartels of graduate students wherein morphologies for galaxy samples reached a few thousands). With surveys such as the Sloan Digital Sky Survey looming on the horizon, a new approach would be necessary in order to take advantage of the unique insights provided by galaxy morphologies.

This necessitated the birth of the Galaxy Zoo project, the first effort to crowd-source the task of galaxy morphology assignment to the general public. Blah blah blah Galaxy Zoo blah. 

While the Galaxy Zoo project has provided a solution that scales visual classification for current surveys  by harnessing the combined power of thousands of volunteers \citep{Lintott2008, Lintott2011, Willett2013, Willett2017, Simmons2017},  producing a prolific amount of scientific output \citep[e.g.,][]{Land2008, Bamford2009, Darg2010, Schawinski2014, Galloway2015, Smethurst2016}; upcoming surveys such as \textit{LSST} and \textit{Euclid} will require a different approach, imaging more than a billion new galaxies  \citep{LSST, Euclid}.  If detailed morphologies can be extracted for just  0.1\% of this imaging, we will have millions of images to contend with. A project of this magnitude would take more than sixty years to classify at Galaxy Zoo's current rate and configuration. Standard visual morphology 
methods will thus be unable to cope with the scale of data. 


\subsection{Automated classifications}
Another approach has been the automated extraction of galaxy morpholgies with the development of both parametric and non-parametric structural indicators. 

It is well known that a galaxy's light profile can be modelled according to the function [insert function here] where the Sersic index, $n$, has been shown to correlate strongly with a distinction between early- and late-type galaxies. In particular, a Sersic index of $n=4$ corresponds to a de Vaucouleurs' profile which well describes the surface brightness of an elliptical galaxy as a function of its apparent radius from the galaxy's center. On the other hand, a Sersic index of $n=1$ describes an exponential profile which is a good description for spiral galaxies. 

A drawback to the parametric approach is the need to assume the underlying distribution and while this works technique works well for galaxies that are obviously elliptical or spiral, it produces mixed results for other morphological types, i.e., irregulars or peculiars, which have low central concentration resulting in a low Sersic index, but which do not have disks or spiral arms. 

Non-paramteric structural indicators require fewer assumptions on the data and are instead derived observationally. Several of these diagnostics have been developed over the past couple decades, each probing a different part of the galaxy's light profile and thus its overall dynamical distribution. The most common diagnostics are described below. 

Closely related to the Sersic index is the non-paramatric diagnostic of concentration. A galaxy's concentration aims to identify how dense a galaxy's central surface brightness profile is. Concentration, originally conceived by Abraham(?) has had several definitions over the years. The most common consider the ratio of the aggregated light within two concentric apertures about the galaxy's center. Typically, these apertures contain 50 and 90\% of the galaxy's light (another common approach is to use 50\% and 80\%). Because 


Peng(?) [was that the first?] developed the first model to decompose a galaxy's light profile into 

Another approach has been the automated extraction of morphologies with the development of parametric \citep{Sersic1968, Odewahn2002, Peng2002}, and non-parametric 
\citep{Abraham1994, 
	   Conselice2003, 
	   Abraham2003, 
	   Lotz2004,  
	   Freeman2013} 
structural indicators. While these scale well to large samples 
\citep[e.g.,][]{Simard2011, 
			Griffith2012, 
			Casteels2014, 
			Holwerda2014, 
			Meert2016}, 
they often fail to capture detailed structure and can provide only statistical morphologies with large uncertainties \cite[e.g.,][]{Abraham1996, Bershady2000}. 


\subsection{Machine learning}
Machine learning techniques are becoming increasingly popular for classification and image processing tasks. Another automated approach, these generally work by defining a set of features that describe the morphology in an $N$-dimensional space. The location in this morphology space defines a morphological type for each galaxy. Learning the morphology space can be achieved through algorithms such as Support Vector Machines \citep{HuertasCompany2008} or Principal Component Analysis \citep{Watanabe1985, Scarlata2007}. Another approach is through deep learning, a machine learning technique that attempts to model high level abstractions. Algorithms like convolutional and artificial neural networks (CNNs, ANNs) have been used for galaxy morphology classification with impressive accuracy 
\citep{Ball2004, 
	Banerji2010, 
	Dieleman2015, 
	HuertasCompany2015}. 
A drawback to all machine learning classification techniques is the need for standardized training data, with more complex algorithms requiring more data. Furthermore, these data must be consistent for each survey: differences in resolution and depth can be implicitly learned by the algorithm making their application to disparate surveys challenging. 


\section{Overview of Galaxy Zoo: Express}

 In this work we present a system that preserves the best features of both visual and automatic classifications, developing for the first time a framework that brings both human and machine intelligence to the task of galaxy morphology to handle the scale and scope of next generation data. We demonstrate the effectiveness of such a system through a re-analysis of visual galaxy morphology classifications collected during the Galaxy Zoo 2 project, and combine these with a Random Forest machine learning algorithm that trains on a suite of non-parametric morphology indicators widely used for automated morphologies. In this paper we focus on the first question of the Galaxy Zoo decision tree. We demonstrate that our method provides a factor of 11.4 increase in the rate of galaxy morphology classification  while maintaining at least 93.5\% classification accuracy as compared to Galaxy Zoo 2 published data. We first present an overview of our framework, which also serves as a blueprint for this paper. 


%%%-------------------------------------------------------
%%% FIGURE:     GZ EXPRESS Schematic
%%%-------------------------------------------------------
\begin{figure*}[ht!]
%\figurenum{1}
\plotone{Figures/human_machine/f1.pdf}
\caption[Schematic of the Galaxy Zoo: Express human+machine hybrid system.]{Schematic of our hybrid system. Humans provide classifications of galaxy images via a web interface. We simulate this with the Galaxy Zoo 2 classification data described in Chapter~\ref{chap:2}. Human classifications are processed with an algorithm described in Chapter~\ref{chap:3}. Subjects that pass a set of thresholds are considered human-retired (fully classified) and provide the training sample for the machine classifier as described in Chapter \ref{chap:4}. The trained machine is applied to all subjects not yet retired. Those that pass an analogous set of machine-specific thresholds are considered machine-retired. The rest remain in the system to be classified by either human or machine. This procedure is repeated  nightly. \label{fig: schematic}}
\end{figure*}


%%----------------------------------------------------------------------------------------------------------------------------------------------------
%%   GALAXY ZOO EXPRESS OVERVIEW
%%---------------------------------------------------------------------------------------------------------------------------------------------------

The Galaxy Zoo Express (GZX) framework combines human and machine to increase morphological classification efficiency, both in terms of the classification rate and required human effort. Figure~\ref{fig: schematic} presents a schematic of GZX including section numbers as a shortcut for the reader. We note that transparent portions  of the schematic represent areas of future work which we explore in Chapter \ref{chap:summary}. Any system combining human and machine classifications will have a set of generic features: a group of human classifiers, at least one machine classifier, and a decision engine which determines how these classifications should be combined.

In this work we demonstrate our system through a re-analysis of  Galaxy Zoo 2 (GZ2) classifications. This allows us to  create simulations of human classifiers. These classifications are used most effectively when processed with SWAP, a Bayesian code described in Chapter \ref{chap:3}, first developed for the Space Warps gravitational lens discovery project~\citep{Marshall2016}. These subjects provide the machine's training sample. 

In Chapter~\ref{chap:4}, we incorporate a machine classifier. We have developed a Random Forest algorithm that trains on measured morphology indicators such as Concentration, Asymmetry, Gini coefficient and \M{20} well-suited for the top-level question of the GZ2 decision tree, discussed in Chapter \ref{chap:2}. After a sufficient number of subjects have been classified by humans, the machine is trained and its performance assessed through cross-validation. This procedure is repeated nightly and the machine's performance increases with size of the training sample, albeit with a performance limit. Once the machine reaches an acceptable level of performance it is applied to the remaining galaxy sample. 

Even with this simple description, one can see that the classification process will progress in three phases.  First, the machine will not yet have reached an acceptable level of performance; only humans contribute to subject classification. Second, the machine's performance will improve; both humans and machine will be responsible for classification. Finally, machine performance will slow; remaining images will likely need to be classified by humans. This blueprint allows even modest machine learning routines to make significant contributions alongside human classifiers and removes the need for ever-increasing performance in machine classification.