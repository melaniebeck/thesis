
% APPENDIX-- ADDITIONAL MIRAC4 IMAGES
\chapter{Additional MIRAC4 Observations} 
\label{app:sofia_mirac}

The observing program with MIRAC included several late type (K through M) giants and supergiants in addition to the observations of $\mu$ Cep and IRC +10420 presented in Chapter 4 (see Table 4.2).  The observing strategy and data reduction procedure was the same for the MIRAC4 images of IRC +10420 as discussed in \S 4.2.2.  The data (previously unpublished) have been re-reduced and are presented in this Appendix.  

Most of the targets have strong 10 $\micron$ silicate emission features in their spectra (seen in ISO-SWS and/or ground-based observations, e.g., \citet{Sylvester:1998}).  Of the seven additional targets imaged at high angular resolution with MIRAC4 in the 8 $-$ 12 $\micron$ range, extended emission is seen in five of the sources (S Per, T Per, RW Cep, BD +24 3902, and RW Cyg), while two did not show noticeable extension ($\rho$ Cas and W Per).  

\subsubsection{S Per: Extended Emission Similar to $\mu$ Cep}
The RSG S Per (Sp. Type M3-4e Ia) is an OH/IR source and a member of the Per OB1 association \citep{Humphreys:1978}, with a distance of 2.4 $\pm$ 0.1 kpc as determined by VLBI H$_{2}$O maser astrometry \citep{Asaki:2010}.  \citet{Schuster:2006} presented \emph{HST} images showing that the star is embedded in an elongated circumstellar envelope with a position angle of $\sim$ 20$\degr$ E of N with a FWHM of $\sim$ 0$\farcs$1 (240 AU).  Those authors noted that the extent and orientation of this envelope match the regions of H$_{2}$O maser emission found by \citet{Richards:1999}, and that it could be explained by bipolarity in the ejecta or a flattened circumstellar halo.  The MIRAC4 images at 8.9 and 9.8 $\micron$ presented in Figure \ref{fig:SPer_MIRAC4} show an extended shape as well, similar to that seen in the case of $\mu$ Cep (compare Figure \ref{muCep_MIRAC3}).  Fitting elliptical Gaussians to S Per's MIRAC4 images yields a mean position angle of 19$\degr$ $\pm$ 2$\degr$ E of N, matching very closely the orientation seen in the $HST$ images.  

% Figure:  S Per
\begin{figure}[htp]
\centering
\includegraphics[scale=0.6]{Figures/addl_mirac_figs/SPer_vs_PSF.png}	
\caption[MIRAC4 images of S Per.]{-- MIRAC4 images of S Per before and after PSF subtraction.  The directions to North and East are indicated.  The pixel scale is 0.037$\arcsec$ pix$^{-1}$.  The dashed white line at position angle 20$\degr$ E of N shows the approximate orientation of the extended envelope seen in the \emph{HST} visual images (see Figure 8 of \citet{Schuster:2006}), which coincides with the direction of extension seen with MIRAC.} 
\label{fig:SPer_MIRAC4}
\end{figure} 

\subsubsection{Sources with Extended Emission Similar to IRC +10420:  T Per, RW Cep, BD +24 3902, and RW Cyg}
Four RSG sources appear extended with circularly symmetric emission:  T Per, RW Cep, BD +24 3902 and RW Cyg.  Their MIRAC4 images are displayed in Figures \ref{fig:TPer_MIRAC4},  \ref{fig:RWCep_MIRAC4},  \ref{fig:BD+243902_MIRAC4}, and  \ref{fig:RWCyg_MIRAC4} respectively.    T Per is an M2 Iab RSG belonging to the Per OB1 association \citep{Humphreys:1978}.  RW Cep is an irregular variable RSG with its reported spectral type ranging from M0 0-Ia \citep{Merrill:1956} to K5 0-Ia \citep{Humphreys:1988} to K2 0-Ia \citep{Rayner:2009}.  BD +24 3902 (Sp. Type M1 Ia) is an RSG member of the Vul OB 1 association, and RW Cyg (Sp. Type M3-4 Ia-Iab) is an RSG member of the Cyg OB 9 association \citep{Humphreys:1988}.  The extended emission seen in varying degrees in these four sources resembles that of IRC +10420 (compare Figure \ref{IRC_MIRAC4}).   As shown in the case of IRC +10420 in \S 4.3.4 and Figure \ref{IRC_M119profile}, such broad profiles may indicate variation in the recent mass-loss history of these stars.  




% Figure:  T Per
\begin{figure}[htp]
\centering
\includegraphics[scale=0.6]{Figures/addl_mirac_figs/TPer_vs_PSF.png}	
\caption[MIRAC4 images of T Per.]{-- MIRAC4 images of T Per  before and after PSF subtraction.  The directions to North and East are indicated.  The pixel scale is 0.027$\arcsec$ pix$^{-1}$.} 
\label{fig:TPer_MIRAC4}
\end{figure} 

% Figure:  RW Cep
\begin{figure}[htp]
\centering
\includegraphics[scale=0.45]{Figures/addl_mirac_figs/RWCep_vs_PSF.png}	
\caption[MIRAC4 images of RW Cep.]{-- MIRAC4 images of RW Cep  before and after PSF subtraction.  The directions to North and East are indicated.  The pixel scale is 0.027$\arcsec$ pix$^{-1}$.} 
\label{fig:RWCep_MIRAC4}
\end{figure} 

% Figure:  BD +24 3902
\begin{figure}[htp]
\centering
\includegraphics[scale=0.45]{Figures/addl_mirac_figs/BD+243902_vs_PSF.png}	
\caption[MIRAC4 images of BD +24 3902.]{-- MIRAC4 images of BD +24 3902  before and after PSF subtraction.  The directions to North and East are indicated.  The pixel scale is 0.027$\arcsec$ pix$^{-1}$.} 
\label{fig:BD+243902_MIRAC4}
\end{figure} 


% Figure:  RW Cyg
\begin{figure}[htp]
\centering
\includegraphics[scale=0.45]{Figures/addl_mirac_figs/RWCyg_vs_PSF.png}	
\caption[MIRAC4 images of RW Cyg.]{-- MIRAC4 images of RW Cyg  before and after PSF subtraction.  The directions to North and East are indicated.  The pixel scale is 0.027$\arcsec$ pix$^{-1}$.} 
\label{fig:RWCyg_MIRAC4}
\end{figure} 



\subsubsection{Sources With Little or No Apparent Extension:  $\rho$ Cas \& W Per}
Finally, included for comparison are the MIRAC4 images of two sources which did not reveal noticeable extended emission: the yellow hypergiant $\rho$ Cas and the RSG W Per, displayed in Figures \ref{fig:rhoCas_MIRAC4} and \ref{fig:WPer_MIRAC4}

% Figure:  rho Cas
\begin{figure}[htp]
\centering
\includegraphics[scale=0.5]{Figures/addl_mirac_figs/RhoCas_vs_PSF.png}	
\caption[MIRAC4 images of $\rho$ Cas.]{-- MIRAC4 images of $\rho$ Cas before and after PSF subtraction.  The directions to North and East are indicated.  The pixel scale is 0.037$\arcsec$ pix$^{-1}$.} 
\label{fig:rhoCas_MIRAC4}
\end{figure} 

% Figure:  W Per
\begin{figure}[htp]
\centering
\includegraphics[scale=0.45]{Figures/addl_mirac_figs/WPer_vs_PSF.png}	
\caption[MIRAC4 images of W Per.]{-- MIRAC4 images of W Per  before and after PSF subtraction.  The directions to North and East are indicated.  The pixel scale is 0.027$\arcsec$ pix$^{-1}$.} 
\label{fig:WPer_MIRAC4}
\end{figure} 





